\usepackage{geometry}
\geometry{
  a4paper,
  lmargin=22mm,
  rmargin=22mm,
  tmargin=25mm,
  bmargin=22mm
}

\usepackage[framemethod=default]{mdframed}
\usepackage{xcolor}

% 色の定義
\definecolor{codegray}{rgb}{0.95, 0.95, 0.95}
\definecolor{bordergray}{rgb}{0.6, 0.6, 0.6}

% --- エラー回避ロジック ---
% 1. Shaded環境が未定義の場合は、空の環境を定義する
% 2. 定義済みの場合は、その内容をクリア(relax)して再定義に備える
\makeatletter
\@ifundefined{Shaded}{
  \newenvironment{Shaded}{}{}
}{
  \let\Shaded\relax
  \let\endShaded\relax
}
\makeatother

% 自分の好きなデザインで「Shaded」を確定させる
\newenvironment{Shaded}{
  \begin{mdframed}[
    linecolor=bordergray,
    backgroundcolor=codegray,
    linewidth=0.4pt,
    roundcorner=0pt,
    innertopmargin=5pt,
    innerbottommargin=5pt,
    innerleftmargin=5pt,
    innerrightmargin=5pt,
    skipabove=\medskipamount,
    skipbelow=\medskipamount,
    splitbottomskip=5pt,
    splittopskip=5pt
  ]
}{
  \end{mdframed}
}
