\usepackage{geometry}
\geometry{
  a4paper,         % 用紙サイズ (デフォルトはletterpaper、A4を明示的に指定)
  lmargin=22mm,    % 左余白
  rmargin=22mm,    % 右余白
  tmargin=25mm,    % 上余白 (例)
  bmargin=22mm     % 下余白 (例)
}

\usepackage[framemethod=default]{mdframed}
\usepackage{xcolor}

% 色の定義
\definecolor{codegray}{rgb}{0.95, 0.95, 0.95}
\definecolor{bordergray}{rgb}{0.6, 0.6, 0.6}

% Pandocがコードブロックに使用する「Shaded」環境を再定義
\renewenvironment{Shaded}{
  \begin{mdframed}[
    linecolor=bordergray,    % 枠線の色
    backgroundcolor=codegray, % 背景色
    linewidth=0.4pt,          % 枠線の太さ
    roundcorner=0pt,          % 角を丸めない(直角)
    innertopmargin=5pt,       % 中の余白(上)
    innerbottommargin=5pt,    % 中の余白(下)
    innerleftmargin=5pt,      % 中の余白(左)
    innerrightmargin=5pt,     % 中の余白(右)
    skipabove=\medskipamount, % ブロック上の余白
    skipbelow=\medskipamount, % ブロック下の余白
    splitbottomskip=5pt,      % ページをまたぐ際のパディング
    splittopskip=5pt
  ]
}{
  \end{mdframed}
}
