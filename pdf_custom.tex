\usepackage{geometry}
\geometry{
  a4paper,
  lmargin=22mm,
  rmargin=22mm,
  tmargin=25mm,
  bmargin=22mm
}

\usepackage[framemethod=default]{mdframed}
\usepackage{xcolor}

% 色の定義
\definecolor{codegray}{rgb}{0.95, 0.95, 0.95}
\definecolor{bordergray}{rgb}{0.6, 0.6, 0.6}

% --- 安全な再定義ロジック ---
\makeatletter
% 1. まずShaded環境が「未定義」の場合のみ、空の環境を作っておく(エラー回避)
\@ifundefined{Shaded}{\newenvironment{Shaded}{}{}}{}
% 画像のデフォルト幅を「本来のサイズ」にする設定
\def\maxwidth{\Gin@nat@width}
\def\maxheight{\Gin@nat@height}
\setkeys{Gin}{width=\Gin@nat@width,keepaspectratio}
\makeatother

% 2. 確実に存在する状態になったので、改めて「renewenvironment」で上書きする
\renewenvironment{Shaded}{
  \begin{mdframed}[
    linecolor=bordergray,
    backgroundcolor=codegray,
    linewidth=0.4pt,
    roundcorner=0pt,
    innertopmargin=5pt,
    innerbottommargin=5pt,
    innerleftmargin=5pt,
    innerrightmargin=5pt,
    skipabove=\medskipamount,
    skipbelow=\medskipamount,
    splitbottomskip=5pt,
    splittopskip=5pt
  ]
}{
  \end{mdframed}
}
